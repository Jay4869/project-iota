\subsection{ANOVA Test}

First of all, ideally, since the block design is highly correlated to trial 
design, we might get the collinearity issue in our model, we would like to check
that and keep only either one design that would be good enough for our generalized 
linear regression model. To address this question, we apply ANOVA test to figure which
design will be the best candidate. In the null hypothesis, we would like to test 
$\widehat{\beta_{block}}$ = 0 or $\widehat{\beta_{trial}}$ = 0, and we would prefer the 
model that has the lowest residual sum of square (RSS). The RSSs we obtained from 
the ANOVA test are 253183.7, 263744.3, 260204.09 for mixed design, model without 
block design, and model without trial design, respectively. Since he mixed design model 
has the lowest RSS, we decide to choose the mixed model to do further analysis.

\subsection{Cross Validation}

Since we have already tested the mixed design model with ANOVA, we would like to 
apply a 5-folds cross validation (CV) to select a model from the mixed models with 
and without discrete cosine transformation (DCT), and we are going to select a model
base on cross validation score, calculated as the mean of squared errors (MSE). 
Mean of squared errors (mixed design no dct): 71461.0961108, 
Mean of squared errors (mixed design with dct): 191332664.585. 
Therefore, the final model we select is the mixed design model without DCT 
because its MSE is the lower one. 

\subsection{Comparing 2-back and 0-back}

Finally, in the experiment design, we know the 2-back test asks for two letters back, 
while 0-back asks for a specified letter. We are interested in find the voxels that responded 
differently when performing these two tasks. We fitted our final design models to 
both 0-back and 2-back task, and obtained $\hat{\beta}$ from each model. We set the null 
hypothesis test for t-test to be $\widehat{\beta_{0-back}}$ - $\widehat{\beta_{2-back}}$ = 0. 
We reject the null hypothesis test if the p value is lower than 0.05/133 from Bonferroni 
correction. We found only 108 voxels that rejected the null hypothesis, meaning that these 
voxels in the brain responded differently when performing 0-back and 2-back tasks. 

