\documentclass[11pt]{article}

\usepackage[margin=0.75in]{geometry}

\title{Working Memory Related Brain Network Connectivity in Individuals with Schizophrenia and Their Siblings}
\author{
  Li, Jie\\
  \texttt{Jay4869}
  \and
  Li, Zeyu\\
  \texttt{lizeyuyuz}
  \and
  Yun, Chuan\\
  \texttt{ay2456}
  \and
  Zhang, Qiangyuan\\
  \texttt{amandazhang}
}

\bibliographystyle{siam}

\begin{document}
\maketitle

\abstract{A recent study shows that schizophrenia reflects a ``dysconnection'' syndrome, which means schizophrenia can result in damaging functional brain connectivity in neural networks. We are interested in schizophrenia, so we decide to look into their studies to see if their conclusion makes sense. Therefore, we download the data they got during their tests and redo their analysis by following their steps in the paper. Even though we aren't able to go through all the analysis they did, we expect to get the same result for the analysis we do.}

\section{Introduction}

Schizophrenia is a chronic, severe, and disabling brain disorder that has affected people throughout history. Or you can just think it as an illness that causes abnormal social behavior and failure to recognize what is real. Previous studies showed that changes in the function of a single brain region, or even a brain system, cannot explain the functional impairments seen in this illness. This means that the causes of schizophrenia is much more complicated than we think. A recent study shows that individuals with schizophrenia have reduced connectivity between neural networks, which could be a forward step to solve the cause of this illness.

The paper suggests this conclusion is ``Working memory related brain network connectivity in individuals with schizophrenia and their siblings'' \cite{schizophreniabrainconnectivity}, and the data they used is available on OpenFMRI.org under ``Working memory in healthy and schizophrenic individuals''. The paper thinks that neural networks which are critical for cognitive function might be affected by schizophrenia, because individuals with schizophrenia have severe cognitive issues. In particular, these neural networks are (1) Dorsal fronto-parietal network (FP), (2) Cingulo-opercular network (CO), (3) Cerebellar network (CER) and (4) "Default mode" network (DMN). 

You should explain the basic idea of the
paper in a paragraph.  You should also perform basic sanity check on the data
(e.g., can you downloaded, can you load the files, confirm that you have the
correct number of subjects).

Briefly explain what reproducibility means and in what sense you will
try to reproduce this study.

\section{Data}

\section{Methods}
\section{Results}
\section{Discussion}


\bibliography{project}

\end{document}
