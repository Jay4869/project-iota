\documentclass[11pt]{article} \usepackage{graphicx} \graphicspath{ {images/} }
\usepackage[margin=0.75in]{geometry}

\title{Reproducing fMRI Data Analysis on Brain Connectivity} \author{ Li, Jie\\
  \texttt{Jay4869} \and Li, Zeyu\\ \texttt{lizeyuyuz} \and Yun, Chuan\\
  \texttt{ay2456} \and Zhang, Qiangyuan\\ \texttt{amandazhang} }

\bibliographystyle{siam}

\begin{document} \maketitle

\abstract{In this research paper, we analyse the fMRI data from the study done
  by Repov and Barch exploring the relationship between schizophrenia and brain
  connectivity \cite{repovs_barch1, repovs_barch2}. Instead of reproducing
  ANOVA analysis of within and between brain network connectivity, we focus on
analysing fMRI scans of one subject using generalized linear model. Before
data analysis, we smoothed the data spatially and corrected noise in the fMRI
data. On the preprocessed data, we perform student t test to identify the voxels
responding to designed stimulus. Having localized the activated voxels, we
closely examine the behaviour of these voxels and validated our assumptions.}

\section{Introduction}

\par The paper of which our research is based is titled ``Working memory related
brain network connectivity in individuals with schizophrenia and their
siblings'' \cite{repovs_barch1, repovs_barch2}. Schizophrenia is a chronic,
severe, and disabling brain disorder. Previous studies have shown that changes
in the function of a single brain region, or even a brain system, cannot explain
the functional impairments seen in this illness \ \cite{repovs_barch1}.
However, Repovs and Barch attempts to show that individuals with schizophrenia
have reduced connectivity within and between neural networks, which could be a
forward step to understanding schizophrenia. 

Repov and Barch systematically examine changes in functional connectivity across
rest and different task states in order to make an inference on characteristics
of schizophrenia. They asseess connectivity using blood oxygen level dependent
(BOLD) time series acquired using fMRI. They find four types of participates, individuals with
schizophrenia, the siblings of individuals with schizophrenia, healthy controls
and the siblings of healthy controls. They designed three working memory loads
of an N-back task and designated four regions of interest (ROIs). The four brain
networks are: (1) Dorsal fronto-parietal network (FP), (2) Cingulo-opercular
network (CO), (3) Cerebellar network (CER) and (4) "Default mode" network (DMN).
The objective of the study is to examine the altered functional connectivity
within and between these four brain networks when the participant perform a
designated N-back task. With ANOVA analysis, it is found that individuals with
schizophrenia and their siblings show consistent reductions in connectivity
between both the FP and CO networks with the CER network. 

We narrow our focus and analysis on a single schizophrenic
subject. The fMRI data in the study shows the changes in
blood flow in the brain, and analysing this data can help us understand how
schizophrenic brains respond during task performance. We detail in the following the steps
we take to analyse the data. We attempted both spatial and temporal smoothing on
the voxels in order to reduce noise. We also performed spectral analysis on fMRI data to
correct noise in the data. After preprocessing the data, we fit
multiple linear regression model using two different sets of experiment
conditions. Having obtained coefficient estimates, we performed Student t-tests
on each voxel to examine whether the particular voxel has show significant
activity subject to appropriate thresholds. Performing t-tests relying on a set
of assumptions that we need to validate. We performed the Shaprio-Wilk test to
check the normality assumption.

\section{Data}

The data used in their published paper is available on the website OpenFMRI.org.
The are many data files grouped by subjects, but we will be focusing on only one
schizophrenic subject--subject 001, who is male, Caucasian and Schizophrenic.
The data has four dimensions (91 x 109 x 91 x 137), consisting of three dimensional voxels that are moving
across time. We removed the first four outlying data along the time dimension,
which leaves us 133 images. (Need to add how to come to understand the BOLD
signal = signal + noise, and the necessity of remove noise)

\section{Linear modelling methods} 

The basic idea of linear models is to find the relationship between a response
variable and one or more explanatory variables (regressors). For our purposes,
we attempt to define a relationship between the BOLD signals and the working
memory tasks. Mathematically, our model is as follows:
 $$ y = X \beta + \epsilon, X = \{1, x_1, x_2, ..., x_k\}$$ To perform
 hypothesis tests using this model, we assume linearity of the relationship, and
 $E [\epsilon | X ] = 0$, $Var [ \epsilon | X ] = \Omega$. We solve for linear
 regression using generalized least square: $\hat{\beta} = (X^T \Omega^{-1}
 X)^{-1} X^T \Omega^{-1} Y$. After finding estimates for the $\beta$s, we perform Student
 t-test for each $\beta_i$, where $H_0: \beta_i = 0, i = \{1, 2, ..., k\}$.

\subsection{Block design}
\subsubsection{Convolution and design matrix}
We need a good design matrix, containing data on the explanatory variables, to
be used in our linear models to explain as much variation in the observed fMRI
data as possible. We studied the design of the experiment closely. fMRI scans were
acquired while participants perform specified memory tasks. They are to respond
for each letter shown whether it was the same as a pre-specified letter
(0-back), the same as the immediately preceding letter (1-back), or the same as
the letter shown two trials previously (2-back). For our subject, there were
three BOLD runs, each consisting of two blocks of 0-back, 1-back, or 2-back
working memory task. 

We examined the seven study conditions closely. The conditions are unusual from
what we are familiar with in that they are consisted of fractions of durations
and negative amplitudes. The first condition consists of
the start-cues, the second condition consists of the task targets (amplitudes
are all one), the third condition consists of the task non-target (the positive amplitude indicates
correct identification of the target and the negative amplitude indicates the
non-response to the non-target), the fourth condition consists of end-cues, and
the fifth condition consists of durations of the two task blocks. There is no
information on what the sixth file mean, and the seventh condition consists of
the erroneous responses.

Since we were only familiar with convolution for block design, we initially used
only the condition one, four and five, which give us the start time, the
duration, and end time of the two task blocks. Due to the delay of signal after
the onset of neural activity, we make use of the hemodynamic response
function (HRF). The idea is to model the signal response as the convolution of
the stimulus function with the HRF. Convolving with stimulus timing allows us to
get idealized response as an explanatory variable (a regressor included in
design matrix). We bear in mind the limitations of this
design matrix due to its omission of trials within each of two task blocks. 

\subsubsection{Linear regression results}
We included three convolved response in our design matrix along with a constant
term. We obtained three beta estimates for each voxel, and we perform t-test
where null hypothesis claims that $\beta_1 + \beta_2 + \beta_3 = 0$. 

\subsection{Event-related design} 
\subsubsection{Convolution and design matrix}
Previously we used block design that has too long of duration, and therefore
unable to explain much of the variation in the BOLD signals. We revised
convolution of stimulus using a finer and higher resolution on time. With trials
within task blocks convolved and included in the design matrix, we expect the
result from linear regression have higher explanatory power. 

\subsubsection{Data Preprocessing} 
To reduce noise and improve signal in the fMRI data, we attempted smoothing of
the fMRI data.We experimented with smoothing temporally using discrete cosine
transformations as well as Gaussian kernel smoother. We added a set of discrete
cosine transformation basis to the design matrix to filter out the low frequency
noises present in the BOLD signal.

We also spatially smoothed the data. 

\subsection{Generalized linear regression results}
(add)



\subsection{Validation}
We use Shaprio-Wilk's test to validate our normality assumption on the error
term. (More)

\section{Lessons and obstacles}

The hardest part was to connect the theoretical models with fMRI data to make
appropriate inferences. 

\bibliography{report}

\end{document}
